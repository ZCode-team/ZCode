\documentclass[10pt]{article}
\usepackage[]{algorithm2e,listings}
\usepackage[utf8]{inputenc}
\usepackage[francais]{babel}
\usepackage[T1]{fontenc}
\usepackage{url,hyperref}
\begin{document}
\lstset{language=C++}
\section{Raffinements et grossissement, graduation du maillage et
  travail par tâches.}
On travaille dans tous les cas sur les slots d'une SlotCollection. Si
on numérote les slots de la collection de 0 à $n$, on utilise en
général un \texttt{parallel\_for} qui va découper récursivement
l'intervalle $[0,n]$ en sous intervalles $[n_i^1,n_i^2]$ qui
correspondent aux tâches. Le problème est qu'on peut avoir besoin
d'écrire hors de l'intervalle  $[n_i^1,n_i^2]$, ce qui n'est pas
\emph{thread safe}.

On peut résoudre ce problème élégamment avec une réduction en utilisant
un
\href{https://software.intel.com/en-us/node/506155}{\texttt{parallel\_deterministic\_reduce}}
(et un
\href{https://software.intel.com/en-us/node/506154}{\texttt{parallel\_reduce}}).

\subsection{Exemple}On prend l'exemple
du raffinement; chaque tâche travaille sur son intervalle associé
$[n_i^1,n_i^2]$.

\begin{algorithm}[H]
\For{$s\leftarrow n_i^1$ \KwTo $n_i^2$}{
  \For{$\rm{node} \in s$}{
    If($\rm{node}$  and refineMe)
    {
      \\
      Calculer les fils de $\rm{node}$.
      \\
      Stocker les fils.
      }
}
}
\end{algorithm}
Évidemment certains fils peuvent ne pas appartenir à l'intervalle
d'abscisses couvert par les slots de  $[n_i^1,n_i^2]$, et ne peuvent
donc pas être stockés dans les slots de  $[n_i^1,n_i^2]$.

\subsection{Schéma de la classe à utiliser avec
  \texttt{parallel\_deterministic\_reduce}}Pour traiter l'exemple
ci-dessus on utilise une classe (c'est aussi possible avec une lambda
fonction), adaptée aux réductions (donc, entre autres avec une
méthode \texttt{join}).
\begin{itemize}
  \item Définitions, cosntructeur:
\begin{lstlisting}
  class Refine{
    SlotCollection SC;
    slot left,right;
    Node Zmin,Zmax;
    public:
    Refine(SlotCollection S)
    {
      /*  
      faire ce qu'il faut pour initialiser SC et uniquement SC.
      (sans faire de copie, bien sur, SC est donc une reference ou un
      truc dans ce genre).
      */
    }
\end{lstlisting}
\item Le travail de raffinement est effectué dans l'\texttt{operator()}:
\begin{lstlisting}
  void operator()( const blocked_range<int>& r )
  {
    Zmin= SC[r.begin()].zmin; Zmax= SC[r.end()-1].zmax;
    /* on deroule  l'agorithme decrit plus haut */
    /* stockage des nouveaux noeuds ?????*/
  }
\end{lstlisting}

Dans cet opérateur, on stocke les nœuds qui débordent de l'intervalle
de slots dans le slot \texttt{left} ou dans le slot \texttt{right} selon
que leur z est 
inférieur à \texttt{Zmin} ou supérieur à \texttt{Zmax}.
Les noeuds \og normaux\fg{} sont stockés dans le bon slot de
l'intervalle décrit par le \texttt{blocked\_range} \texttt{r}.

\item Le constructeur par copie
\begin{lstlisting}
  Refine(Refine&, split)
\end{lstlisting}
  n'initialise que
\texttt{SC}.

\item Pour la méthode \texttt{join}:
\begin{lstlisting}
  void join(Refine& R)
\end{lstlisting}
on sait que les slots de \texttt{this->right} vont dans \texttt{R.SC} et que ceux de
\texttt{R.left} vont dans \texttt{this->SC}: il suffit donc d'effectuer les
insertions, et de faire en sorte que, à  la fin, \texttt{left} reste  inchangé, tandis que
\texttt{this->right} est \texttt{R.right}.
\end{itemize}

\subsection{Le \texttt{parallel\_reduce}} Si on suit la documentation de
\href{https://software.intel.com/en-us/node/506154}{\texttt{parallel\_reduce}},
la même instantiation de la classe \texttt{Refine} peut traiter
plusieurs \texttt{blocked\_range}, \textbf{sans appeler la méthode
  join}.
Il faut donc adapter la classe (ce doit être possible en testant, quand
on entre dans \texttt{operator()} si les slots \texttt{left} et
\texttt{right} sont vides. Dans le cas où ils ne sont pas vides, il
faut faire les insertions qui peuvent l'être, dans \texttt{operator()}.
\end{document}
